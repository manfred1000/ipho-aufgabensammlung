\ifx \mpreamble \undefined
\documentclass[12pt,a4paper]{article}
\usepackage{answers}
\usepackage{microtype}
\usepackage[left=3cm,top=2cm,bottom=3cm,right=2cm,includehead,includefoot]{geometry}

\usepackage{amsfonts,amsmath,amssymb,amsthm,graphicx}
\usepackage[utf8]{inputenc}
\usepackage[T1]{fontenc}
\usepackage{ngerman}

\usepackage{pstricks}
\usepackage{pst-circ}
\usepackage{pst-plot}
%\usepackage{pst-node}
\usepackage{booktabs}

\ifx \envfinal \empty
\usepackage{pst-pdf}

\usepackage{hyperref}
\hypersetup{
bookmarks=true,
pdfpagemode=UseNone,
pdfstartview=FitH,
pdfdisplaydoctitle=true,
pdflang=de-DE,
pdfborder={0 0 0}, % No link borders
unicode=true,
pdftitle={IPhO-Aufgabensammlung},
pdfauthor={Pavel Zorin},
pdfsubject={Aufgaben der 3. und 4. Runden deutscher Auswahl zur IPhO},
pdfkeywords={}
}

\fi

%Times 10^n
\newcommand{\ee}[1]{\cdot 10^{#1}}
%Units
\newcommand{\unit}[1]{\,\mathrm{#1}}
%Differential d's
\newcommand{\dif}{\mathrm{d}}
\newcommand{\tdif}[2]{\frac{\dif#1}{\dif#2}}
\newcommand{\pdif}[2]{\frac{\partial#1}{\partial#2}}
\newcommand{\ppdif}[2]{\frac{\partial^{2}#1}{\partial#2^{2}}}
%Degree
\newcommand{\degr}{^\circ}
%Degree Celsius (C) symbol
\newcommand{\cel}{\,^\circ\mathrm{C}}
% Hinweis
\newcommand{\hinweis}{\emph{Hinweis:} }
% Aufgaben mit Buchstaben numerieren
\newenvironment{abcenum}{\renewcommand{\labelenumi}{(\alph{enumi})} \begin{enumerate}}{\end{enumerate}\renewcommand{\labelenumi}{\theenumi .}}

\def \mpreamble {}
\else
%\ref{test}
\fi
\ifx \envfinal \undefined


\newcommand{\skizze}[1]{
\begin{figure}
\begin{center}
#1
\end{center}
\end{figure}
}




%\documentclass[12pt,a4paper]{article}
\renewcommand{\thesection}{}
\renewcommand{\thesubsection}{\arabic{subsection}}

\newenvironment{problem}[2]{
\subsection{#1 \emph{(#2 Punkte)}}
}{}
\newenvironment{solution}{\subsection{Lösung}}{}
\newenvironment{expsolution}{\subsection*{Lösung}}{}

\begin{document}

\fi


\begin{problem}{Warme Gluehwendel}{4}
An ein Glühwendel, wird die Spannung $U_1$ angelegt und es hat die Temperatur $T_1$. Bei der doppelten Spannung hat es eine um $\Delta T=30\unit{K}$ höhere Temperatur. Der Widerstand des Glühwendels sei im Relevanten Temperaturbereich konstant. Die Umgebungstemperatur ist $\Theta=20\unit{\degr C}$
\begin{solution}
$$P_1=\frac{U_1^2}R=\sigma A (T_1^4 - T_0^4)$$
$$P_2=\frac{U_2^2}R=4\frac{U_1^2}R=\sigma A ((T_1+\Delta T)^4-T_0^4)$$
$$\frac{P_1}{P_2}=\frac 14=\frac{T_1^4-T_0^4}{(T_1+\Delta T)^4-T_0^4}=\frac{(T_1^2-T_0^2)(T_1^2+T_0^2)}{((T_1+\Delta T)^2-T_0^2)((T_1+\Delta T)^2+T_0^2)}\approx\frac{T_1^2-T_0^2}{(T_1+\Delta T)^2-T_0^2)}$$
$$(T_1+\Delta T)^2-T_0^2=4T_1^2-4T_0^2$$
$$-3T_1^2+2T_1\Delta T+\Delta T^2+3T_0^2=0 \Rightarrow T_1^2-\frac 23 T_1\Delta T-\frac 13 \Delta T^2-T_0^2=0$$
$$(T_1)_{1;2}=\frac 13\Delta T\pm\sqrt{\frac 49\Delta T^2+T_0^2}$$
Da die Temperatur positiv sein muss, folgt: $(T_1)_1\approx 304\unit{K} \Rightarrow (\vartheta_1)_1 \approx 30,7\unit{\degr C}$
Eine gültige Lösungsvariante (ohne wesentlichen Punktabzug) bestand darin, die abgegebene Leistung als proportional zur Temperaturdifferenz zu betrachten:
$$P_1=k(T_1-T_0), P_2=k(T_1+\Delta T-T_0) \Rightarrow T_1+\Delta T-T_0=4T_1-4T_0$$
$$\Rightarrow T_1=\frac 13(\Delta T+3T_0)\approx 303\unit{K}\Rightarrow \vartheta=30\unit{\degr C}$$
\end{solution}
\end{problem}

\begin{problem}{Zerfallsreihe}{5}
In einer Probe sind die beiden Thoriumisotope im natürlichen Gleichgewicht. Nach dem entfernen aller anderen Atome hat die Probe eine Masse von $1 \unit{kg}$. Wie verhält sich die absolute Anzahl $N$ der Radon-220-Atome im Zeitraum von $10^{-3}\unit{a}$ bis $10^3\unit{a}$? Geben Sie den Verlauf so genau wie möglich wieder!
\end{problem}

\begin{problem}{Untergehende Sonne}{4}
Der $1.8\unit{m}$ hohe Indianer Pato betrachtet am Äquator zur Tag- und Nachtgleiche die untergehende Sonne.
 \begin{abcenum}
  \item Als der obere Rand der Sonne gerade untergeht, kommt ihm die Idee auf einen in der nähe gelegenen $4.2 \unit{m}$ hohen Hügel zu steigen um die Sonne ein weiteres Mal untergehen zu sehen. Wie viel Zeit hat er dafür maximal?
  \item Als Pato die Sonne das erste Mal untergehen sieht, sieht der gleichgroße Indianer Taxo von einem Berg aus den unteren Rand der Sonne gerade untergehen. Wie hoch ist dieser Berg, wenn bekannt ist, dass die Sonne einen Sehwinkel von $0.5 \degr$ besitzt?
 \end{abcenum}
\end{problem}

\begin{problem}{Unerwartete Schaltungen}{5}
\skizze{
\begin{pspicture}(-1,0)(5,5)
  \pnode(1,1){phi2}
  \pnode(1,4){phi1}
  \pnode(3,1){A}
  \pnode(3,4){B}
  \pnode(4,1){C}
  \pnode(4,4){D}
  \pnode(4,2.5){S}
  \wire(A)(C)
  \wire(B)(D)
  \wire(B)(phi1)
  \resistor(C)(S){$R_1$}
  \capacitor(A)(B){$C$}
  \resistor(phi2)(A){$R_2$}
  \switch(S)(D){}
  \tension[labeloffset=-0.7](phi1)(phi2){$U_0e^{i\omega t}$}
\end{pspicture}
\begin{pspicture}(-1,0)(5,5)
  \pnode(1,1){phi2}
  \pnode(1,4){phi1}
  \pnode(3,2){A}
  \pnode(4,2){A2}
  \pnode(4,1){B}
  \pnode(4,4){C}
  \pnode(3,4){D}
  \coil(phi2)(B){$L$}
  \capacitor(A)(D){$C$}
  \resistor[labeloffset=-0.6](A2)(C){$R$}
  \tension[labeloffset=-0.7](phi1)(phi2){$U_0e^{i\omega t}$}
  \wire(C)(D)
  \wire(D)(phi1)
  \wire(A2)(A)
  \wire(A2)(B)
\end{pspicture}
}
 \begin{abcenum}
  \item In der linken Schaltung ist der Widerstand $R_1$ so zu wählen, dass die Impedanz der Schaltung bei geöffnetem und geschlossenem Schalter gleich groß wird.
  \item In der rechten Schaltung ist die Kreisfrequenz so zu wählen, dass die Impedanz der Schaltung unabhängig vom Widerstand $R$ wird.
 \end{abcenum}
\end{problem}

\begin{problem}{Meteoritentemperatur}{7}
Ein kleiner kugelförmiger schwarzer Meteorit der Masse $m=10^{-10}\unit{kg}$ fällt durch die Erdatmosphäre.
Die Stöße der Luftmoleküle mit dem Meteoriten sind als vollständig unelastisch zu betrachten, die Massenänderung des Meteoriten aber zu vernachlässigen.
Die Geschwindigkeit in Abhängigkeit der Höhe ist gegeben durch $v(h)=v_0\exp{-\frac{\rho(h)\pi R^2h}{m}}$.
 \begin{abcenum}
  \item Bestimmen Sie die Anzahl der Stöße des Meteoriten mit den Molekülen der Atmosphäre je Sekunde als Funktion der Geschwindigkeit.
  \item Bestimmen Sie die Temperatur des Meteoriten als Funktion der Geschwindigkeit.
  \item Wann erreicht der Meteorit seine höchste Temperatur und wie groß ist diese?
 \end{abcenum}
\begin{solution}
Dies ist eine vorläufige, noch nicht zufriedenstellende Lösung der Aufgabe.
%TODO: Formel überprüfen ... vllt mal mit Hr. Petersen sprechen
\begin{abcenum}
\item $\dot{N}=\frac{\dot{m}}{m_t}=\frac{\rho(h) \pi R^2 v(h)}{m_t}$
\item $P=\sigma 4\pi R^2 T^4=\frac{m_t}2v^2\frac{\rho \pi R^2 v}{m_t}=\frac{mv^3}{2h}\ln{\left(\frac{v_0}{v}\right)} \Rightarrow T=\left(\frac{mv^3\ln{\left(\frac{v_0}{v}\right)}}{3\pi R^2h\sigma}\right)^{\frac 14}$
\item  $0=\frac {\partial}{\partial v}T^4=\frac{m}{8\pi R^2h\sigma}v^2\left(3\ln\left(\frac{v_0}{v}-1\right)\right)$ mit den zwei Bedingungen: $\ln\frac{v}{v_0}=-\frac 13$ und daraus folgend: $\frac{m}{3\pi R^2}=h\rho(h)$ aus der man im speziellen Fall nun eine oder mehrere Lösungen $h_l$ gewinnen muss, woraus folgt: $\hat{T}=\left(\frac{mv_0^3}{24\pi e R^2h_l\sigma}\right)^{\frac 14}$
\end{abcenum}
\end{solution}
\end{problem}

\begin{problem}{Bewegte Glasscheibe}{5}
Licht ist auf dem Weg von einer Lichtquelle A zu einem Empfänger B. Ein Glaskörper der Brechzahl $n$ und der Dicke $D$ in seinem Ruhesystem bewegt sich parallel zu $\overline{AB}$ mit der Geschwindigkeit $v$, sodass das Licht senkrecht in den Glaskörper eintritt. Wie stark beeinflusst die Anwesenheit des Körpers die Laufzeit des Lichts?
\begin{solution}
Im Ruhesystem des Glaskörpers bewegt sich der Lichtstrahl mit $\frac cn$ durch den Glaskörper. Im Laborsystem ist diese Geschwindigkeit (nach der Regel für die Addition von Geschwindigkeiten): $c_n'=c\frac{1+\beta n}{n+\beta}$. Im Laborsystem sei die Länge des Glaskörpes mit $D'$ bezeichnet. Nun berechnen wir die Zeit, die das Licht im Körper benötigt. Dazu ist zu beachten, dass sich der Körper noch mit der Geschwindigkeit $v$ bewegt. Das heißt, es gilt: $c_n'\Delta t_{Glas}=D'+v\Delta t_{Glas}$.

Daraus folgt: $\Delta t_{Glas}=\frac{D'}{c_n'-v}$ und der im Glas zurückgelegte Weg errechnet sich zu: $l_{Glas}=\frac{c_n'D'}{c_n'-v}=D'+\frac{vD'}{c_n'-v}$. Der Rest des Weges $l-l_{Glas}$ wird mit $c$ zurückgelegt. Vergleicht man mit der Laufzeit ohne Glas, so fällt auf, dass das Wesentliche an der Laufzeit, die Laufzeitdifferenz des Lichtweges mit Glas und des Lichtweges ohne Glas ist. Folglich: $p:=\Delta \Delta t=-\frac 1c (\frac{vD'}{c_n'-v}+D')+\frac{D'}{c_n'-v}$. Dies lässt sich wie folgt umformen:
\begin{eqnarray}
\nonumber p&=&-\frac{D'}c+\frac{\beta D'}{c_n'-v}+\frac{D'}{c_n'-v}=\frac{D'}{c}\left(-1+\frac{c}{c\frac{1+\beta n}{n+\beta}-v}(\beta+1)\right)\\
\nonumber &=&\frac{D'}c\left(-1+\frac{n+\beta}{1+\beta n-\beta n-\beta^2}(1-\beta)\right)\\
\nonumber &=&\frac{D\sqrt{(1-\beta)(1+\beta)}}{c}\frac{n-1}{1+\beta}\\
\nonumber &=&\frac Dc(n-1)\sqrt{\frac{1-\beta}{1+\beta}}
\end{eqnarray}
Also beträgt die Gesamtlaufzeit: $\delta t_{ges}=\frac lc+\frac Dc(n-1)\sqrt{\frac{1-\beta}{1+\beta}}$
\end{solution}
\end{problem}

\input{previewending.tex}
