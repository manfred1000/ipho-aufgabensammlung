\ifx \mpreamble \undefined
\documentclass[12pt,a4paper]{article}
\usepackage{answers}
\usepackage{microtype}
\usepackage[left=3cm,top=2cm,bottom=3cm,right=2cm,includehead,includefoot]{geometry}

\usepackage{amsfonts,amsmath,amssymb,amsthm,graphicx}
\usepackage[utf8]{inputenc}
\usepackage[T1]{fontenc}
\usepackage{ngerman}

\usepackage{pstricks}
\usepackage{pst-circ}
\usepackage{pst-plot}
%\usepackage{pst-node}
\usepackage{booktabs}

\ifx \envfinal \empty
\usepackage{pst-pdf}

\usepackage{hyperref}
\hypersetup{
bookmarks=true,
pdfpagemode=UseNone,
pdfstartview=FitH,
pdfdisplaydoctitle=true,
pdflang=de-DE,
pdfborder={0 0 0}, % No link borders
unicode=true,
pdftitle={IPhO-Aufgabensammlung},
pdfauthor={Pavel Zorin},
pdfsubject={Aufgaben der 3. und 4. Runden deutscher Auswahl zur IPhO},
pdfkeywords={}
}

\fi

%Times 10^n
\newcommand{\ee}[1]{\cdot 10^{#1}}
%Units
\newcommand{\unit}[1]{\,\mathrm{#1}}
%Differential d's
\newcommand{\dif}{\mathrm{d}}
\newcommand{\tdif}[2]{\frac{\dif#1}{\dif#2}}
\newcommand{\pdif}[2]{\frac{\partial#1}{\partial#2}}
\newcommand{\ppdif}[2]{\frac{\partial^{2}#1}{\partial#2^{2}}}
%Degree
\newcommand{\degr}{^\circ}
%Degree Celsius (C) symbol
\newcommand{\cel}{\,^\circ\mathrm{C}}
% Hinweis
\newcommand{\hinweis}{\emph{Hinweis:} }
% Aufgaben mit Buchstaben numerieren
\newenvironment{abcenum}{\renewcommand{\labelenumi}{(\alph{enumi})} \begin{enumerate}}{\end{enumerate}\renewcommand{\labelenumi}{\theenumi .}}

\def \mpreamble {}
\else
%\ref{test}
\fi
\ifx \envfinal \undefined


\newcommand{\skizze}[1]{
\begin{figure}
\begin{center}
#1
\end{center}
\end{figure}
}




%\documentclass[12pt,a4paper]{article}
\renewcommand{\thesection}{}
\renewcommand{\thesubsection}{\arabic{subsection}}

\newenvironment{problem}[2]{
\subsection{#1 \emph{(#2 Punkte)}}
}{}
\newenvironment{solution}{\subsection{Lösung}}{}
\newenvironment{expsolution}{\subsection*{Lösung}}{}

\begin{document}

\fi

\begin{problem}{Kosmische Geschwindigkeiten}{3,5}
Als erste kosmische Geschwindigkeit wird die Geschwindigkeit bezeichnet, die ein Körper benötigt, um eine Kreisbahn um die Erde an der Erdoberfläche zu beschreiben.  Die zweite kosmische Geschwindigkeit ist hingegen notwendig, um dem Gravitationsfeld der Erde gerade ganz zu entfliehen.
\begin{abcenum}
  \item Berechnen Sie das Verhältnis der zweiten zur ersten kosmischen Geschwindigkeit.
  \item Welchen Anteil der zweiten kosmischen Geschwindigkeit muss man einem Körper mitteilen, damit er, von der Erde aus startend, nach einem Jahr zur Erde zurückkehrt?
\end{abcenum}

Sie können in dieser Aufgabe den Luftwiderstand und den Einfluss anderer Himmelskörper außer der Erde vernachlässigen.

Folgende Werte können als bekannt vorausgesetzt werden:
\begin{description}
\item[Radius der Erde:] $r_E=6370\unit{km}$
\item[Masse der Erde:] $m_E=5,97\cdot 10^{24}\unit{kg}$
\end{description}
\begin{solution}
  \begin{abcenum}
    \item 1. Kosmische Geschwindigkeit: $mg=mGm_E r_E^{-2}=m\omega^2r_E\Rightarrow \frac{Gm_E}{r_E}=\omega^2r_E^2=v_1^2\Rightarrow v_1=\sqrt{\frac{Gm_E}{r_E}}$.  2. Kosmische Geschwindigkeit: $\frac 12mv_2^2=W_{Hub}=\lim\limits_{b\rightarrow mGm_e\infty}\int\limits_{r_E}^bdx\ \frac{1}{x^2}=mGm_E\lim\limits_{b\rightarrow \infty}\left(\frac 1{r_E}-\frac 1b\right)=mGm_Er_E^{-1}\Rightarrow v=\sqrt{\frac{2Gm_e}{r_E}}$
  \item Sei also die Erde ein Inertialsystem und die Erdrotation folglich nicht vorhanden, dann gilt nach dem 3. Keplerschen Gesetz: $\left(\frac {a_1}{a_2}\right)^2= \left(\frac{T_1}{T_2}\right)^2$ wobei die Indizes zwei verschiedene geschlossene Ellipsenbahnen um die Erde bezeichnen und $T$ die Umlaufdauern und $a$ die großen Halbachsen bedeuten.  Eine Gerade, die beide Brennpunkte verbindet ist eine gestörte Form der Ellipse und sein nun mit Index $1$ behaftet.  Ein Kreis der direkt um den Erdradius geht ist auch eine gestörte Form der Ellipse und sei nun mit Index $2$ behaftet.  Dann ist $a_2=r_E; T_2=2\pi \frac 1{\omega}=2\pi\frac {r_E}{v_1}$, wobei $v$ hier die erste kosmische Geschwindigkeit ist und somit $T_2=\sqrt{4\pi^2\frac{r_E^3}{Gm_E}}$.  $T_1$ soll ein Jahr sein (Beachte es ist eine Näherung, da der andere Brennpunkt mit dem Erdmittelpunkt zusammenfällt).  Dann ergibt sich:
    \begin{equation*}
      h\approx a_1=\sqrt[3]{T_1^2\frac{Gm_E}{4\pi^2 r_E^3}r_E^3}=\sqrt[3]{T_1^2\frac{Gm_E}{4\pi^2}}\approx 2,15\cdot 10^{9}m
    \end{equation*}
  $\frac{v}{v_{2}}=\sqrt{1-\frac {r_E}h}\approx 0,9985$.  $v$ ist also unwesentlich kleiner als $v_2$.  Es geht auch anders indem man im wesentlichen Kepler 3 nachrechnet: $v(x)=\sqrt{\frac 2m(E_{kin;0}+E_{pot;0}(x)-E_{pot}(x))}$, wobei $E_{pot}(r\rightarrow \infty)=0$ sei. Somit ist:
$1\unit{y}=\int\limits_{r_E}^a\frac{dx}{\sqrt{\frac{2m}\left(mGm_E\left(\frac 1x-\frac 1a\right)\right)}}$, wobei $E_{kin;0}+E_{pot;0}=E_{pot}(a)$, wobei $a$ der Ort ist wo der Körper umkehrt, sodass $E_{kin;0}(a)=0$ ist.  Dieses Integral kann mit der Substitution $x=a\sin\phi$ gelöst werden.
  \end{abcenum}
\end{solution}
\end{problem}

\begin{problem}{Negativer Brechungsindex}{5}
Bestimmte Materialien besitzen für einen engen Wellenlängenbereich der elektromagnetischen Strahlung einen negativen Brechungsindex.  Geht ein Lichtstrahl von einem Medium mit einem Brechungsindex $n_1>0$ in ein Medium mit einem Brechungsindex $n_2<0$ über, so gilt weiterhin das Snelliussche Brechungsgesetz:
\begin{equation*}
  n_1\sin{\alpha_1}=n_2\sin{\alpha_2}
\end{equation*}
Allerdings ist $\alpha_2$ dann negativ.
\begin{abcenum}
  \item Ein kleines Objekt befinde sich in einem Abstand $a$ vor einer großen Platte der Dicke $d$, die aus einem Material mit Brechungsindex $-1$ besteht.  Der Brechungsindex des übrigen Raumes sei $1$.

Geben Sie an, wo sich das Bild des Objektes auf der anderen Seite der Platte befindet und welche Eigenschaften dieses Bild besitzt, d.h. geben Sie dessen Vergrößerung an und entscheiden Sie, ob das Bild reell bzw. virtuell, gespiegelt oder rotiert ist.
  \item Ein kleines Objekt befinde sich in einem kartesischen Koordinatensystem bei $(a,-b,0)$, wobei $a$ und $b$ größer $0$ seien.  Der Raum mit $x>0$ und $y>0$ sei mit einem Material des Brechungsindexes $-1$ angefüllt und der Brechungsindex betrage $1$ für den üblichen Raum.

Die nebenstehende Abbildung zeigt einen Schnitt durch den Raum für $z=0$.  Betrachten Sie im Folgenden nur diese Ebene.

Untersuchen Sie die Strahlen, die vom Objekt kommen und aus dem Raum mit negativem Brechungsindex stammen.  Bestimmen Sie die Position, die Vergrößerung und die Eigenschaften des aus diesen Strahlen entstehenden Bildes.  Von welcher Position aus ist dieses Bild zu sehen?
\end{abcenum}
  \begin{solution}
    
  \end{solution}
\end{problem}

\begin{problem}{Masse und Feder}{4}
  Zwei Gewichte liegen auf einer waagerechten reibungsfreien Unterlage, das linke Gewicht $m$ berühre eine Wand, das andere Gewicht der Masse $3m$ ist durch eine Feder und einen Faden mit dem ersten verbunden.  Die Feder ist anfänglich um $4\unit{cm}$ gestaucht.  Dann wird der Faden durchtrennt.  Bestimmen Sie die maximale Dehnung der Feder im darauffolgenden Prozess!
  \begin{solution}
    Erste Phase:  Beschleunigung der Masse $3m$.  Da Zwangsbedingung und 3. Newtonsche Gesetz dazu führt, dass die Masse $m$ an Ort und Stelle verbleibt (Zwangsbedingung wirkt als Impulsreservoir).  Deshalb gibt es eine Beschleunigung des Schwerpunktes.  Die Schwerpunktsenergie ist $\frac 34$ der Gesamtenergie.  Diese Steckt also in einer Translationsenergie.  Für die Schwingung verbleibt also noch ein viertel der Gesamtenergie.  Das heißt $\hat{E}_{\pot}'=\frac 14\hat{E}_{\pot}$, wobei $\hat{E}_\pot'$ die maximale potentielle Energie der Feder nach dem Stoß ist.  Daraus folgt $\hat{l}'^2=\frac 14\hat{l}^2\Rightarrow \hat{l}'=\frac 12 \hat l=2\unit{cm}$.
  \end{solution}
\end{problem}

\begin{problem}{Houston, wir haben ein Problem}{5,5}
An Bord eines Raumschiffes, das sich in weiter Entfernung von der Erde befindet, ist ein Problem aufgetreten.  Die Astronauten berichten, dass plötzlich ein Loch der Größe von $1\unit{mm^2}$ in der Hülle aufgetreten ist, durch welches Luft ausströmt.  Offensichtlich ist nicht nur die Hülle beschädigt, denn direkt nach der Meldung fällt der Funkkontakt aus.
\begin{abcenum}
  \item Bestimmen Sie, wie lange es dauert, bis der Druck im Inneren des Raumschiffes auf die Hälfte des anfänglichen Druckes von $10^5\unit{Pa}$ abgefallen ist.  Nehmen Sie dazu an, dass die Druckänderung im Raumschiff isotherm abläuft und dass die Wand des Raumschiffes dünn ist.
  \item Berechnen Sie, um wie viel sich die Geschwindigkeit des Raumschiffes dabei ändert.
\end{abcenum}
Sie können folgende Werte für ihre Rechnungen verwenden:
\begin{description}
\item[Temperatur im Raumschiff:] $T=20\unit{\cel}$
\item[Masse des Raumschiffes:] $m=2\cdot 10^4\unit{kg}$
\item[Luftvolumen im Raumschiff:] $V=50\unit{m^3}$
\item[Molare Masse von Luft:] $M_{Luft}=29,0\unit{g mol^{-1}}$
\end{description}
\begin{solution}
\begin{abcenum}
\item
  Bernoulli liefert: $v=\sqrt{\frac{2}{\rho(t)}p(t)}$.  Da $pV=nRT=\const$ und $V=\frac{m_G}{\rho}$ ergibt sich: $\frac{p}{\rho}=\frac{RT}{M_{Luft}}=\const$. $V$ bezeichnet hierbei das Volumen des Gases, wobei sich gedacht wird, dass es sich insgesamt isotherm ausdehnt. (Das das außerhalb der Raumsphäre nicht der Fall ist beeinflusst den isothermen Prozess innerhalb der Raumsphäre nicht).  Somit ist $v=\sqrt{2\frac{RT}{M_{Luft}}}=\sqrt{\frac{2kT}{m_{Luft}}}=\const$.  $\dot{p}=\frac{RT}{M_{Luft}}\dot{\rho}=\frac{RT}{M_{Luft}}\frac{\dot{m_i}}{V_K}$, wobei $V_K$ das Volumen der Raumkapsel ist, welches sich nicht ändert.  Das Gas im Raumschiff bewegt sich nicht, dehnt sich langsam aus, lediglich das Gas, dass direkt am Loch ist wird durch den Druckgradienten beschleunigt.  Da sich die Dichte des Gases in der Raumkapsel nur dadurch ändert, das Gas am Loch mit Dichte $\rho(t)$ ausgestoßen wird, muss die Änderung $\rho(t)$ der Gasmenge, die ausgestoßen wird, während des Austoßes derselben nicht berücksichtigt werden.  $\dot{m_i}=-Av\rho(t)=\frac{-AvpM_{Luft}}{RT}=-\frac{ApM_{Luft}}{RT}\sqrt{\frac {2RT}{M_{Luft}}}$.  Somit ergibt sich:

  \begin{align*}
    \dot{p}&=\frac{RT}{V_KM_{Luft}} \left(-\frac{ApM_{Luft}}{RT}\sqrt{\frac {2RT}{M_{Luft}}}\right)=-\frac{A}{V_K}\sqrt{\frac{2RT}{M_{Luft}}}p\\
  p(t)&=p_0\exp\left(-\frac{A}{V_K}\sqrt{\frac{2RT}{M_{Luft}}} t\right)\\
  t_{\frac 12}&=\ln(2)\frac{V_K}{A}\sqrt{\frac {M_{Luft}}{2RT}}
  \end{align*}
\item Es gilt Impulserhaltung: $\Delta m_{Gas} v=\frac {M_{Luft}\Delta pV_K}{RT}\sqrt{\frac{2RT}{M_{Luft}}}=\Delta pV_K\sqrt{\frac{2M_{Luft}}{RT}}\approx m\Delta v\Rightarrow \Delta v=\frac{\Delta pV_K}{m}\sqrt{\frac{2M_{Luft}}{RT}}$.
\end{solution}
\end{problem}

\begin{problem}{Defribrillatoren und Aufwaertswandler}{7}
 Defribrillatoren werden dazu benutzt, den Herzrhythmus eines nicht regelmäßig schlagenden Herzens wiederherzustellen.  Dazu wird gleichzeitig ein großer Anteil der Herzmuskulzellen durch einen Stromschlag elektrisch stimuliert.

Betrachten Sie einen einfachen Defibrillator bestehend aus einem Kondensator, der sich über zwei mit dem Brustkorb des Patienten verbundene Elektroden in einem Zeitraum von etwa $150\unit{ms}$ entlädt.  Der Widerstand des Brustkorbes zwischen den Elektroden betrage etwa $100\unit{\Omega}$ und die für die Defibrillation notwendige Energie betrage $200\unit{J}$.

\begin{abcenum}
\item Schätzen Sie ab, welche Kapazität der Kondensator besitzen muss und auf welche Spannung er für den Betrieb mindestens aufgeladen werden muss.
\end{abcenum}

Bei mobilen Defibrillatoren, wie sie an einigen öffentlichen Plätzen zu finden sind, wird der Kondensator über eine Batterie aufgeladen.  Da die Spannung $U_=$ der Batterie geringer ist, als die notwendige Kondensatorspannung, muss sie hochgewandelt werden.  Eine Möglichkeit dazu bietet ein so genannter Aufwärtswandler, wie er nebenstehend skizziert ist.

Der Schlater $S$ öffnet und schließt sich periodisch, wobei er einen Anteil $g$ einer sehr kurzen Periodendauer geschlossen und einen Anteil $1-g$ offen ist.  $g$ wird in diesem Fall als Tastverhältnis bezeichnet.  Der eingezeichnete Widerstand soll sehr groß sein und das ohmsche Verhalten des Kondensators abbilden.  Nehmen Sie außerdem an, dass die Diode in Sperrrichtung vollständig sperrt und in Durchlassrichtung keinen Spannungsabfall verursacht.

\begin{abcenum}
  \item Leiten Sie einen Ausdruck für die maximal erreichbare Kondensatorspannung in Abhängigkeit von den auftretenden Größen her.  Alle Bauteile dürfen als ideal angenommen werden.
\item Bestimmen Sie, wie groß das Tastverhältnis gewählt werden muss, um einen Kondensator einer Kapazität von $100\unit{\micro F}$ mit einer $12\unit{V}$ Batterie auf eine Spannung von $500\unit{V}$ aufzuladen, wenn die Induktivität der Spule $5\unit{mH}$ beträgt.
\end{abcenum}

\end{problem}

\begin{problem}{Ein seltsamer Anblick}{5}
Ein Raumschiff bewegt sich mit einer hohen Geschwindigkeit $v$ senkrecht zur Ebene der Milchstraße.  Aus Sicht eines Beobachters in der Milchstraße befindet er sich genau über dem galaktischen Zentrum in einer Entfernung $d$ zur galaktischen Ebene.

Ein Beobachter in dem Raumschiff stellt erstaunt fest, dass er einen Teil der Galaxis nicht hinter, sondern vor sich sieht.
\begin{abcenum}
 \item Untersuchen Sie dieses Phänomen, indem sie berechnen, unter welchem Winkel der Beobachter im Raumschiff einen Lichtstrahl empfängt, der für einen Beobachter in der Galaxis einen Winkel $\theta$ mit der Bahn des Raumschiffes einschließt.  Geben Sie den aus dem Raumschiff beobachteten Winkel $\theta'$ als Funktion des Winkels $\theta$ an.
 \item Die Milchstraße besitzt eine Ausdehnung von etwa $100000\unit{ly}$ und das Raumschiff befindet sich in einer Entfernung von $d=20000\unit{ly}$ von ihr entfernt.  Bestimmen Sie die Geschwindigkeit, mit der sich das Raumschiff mindestens bewegen muss, wenn der Rand der Galaxie unter einem Winkel von $\theta'=150\unit{\cel}$ aus dem Raumschiff zu sehen ist.
\end{abcenum}
\begin{solution}
 Sei $\Sigma'$ das Bezugssystem des Raumschiffes und $\Sigma$ das Bezugssystem der Milchstraße.  Beide werden als Inertialsysteme behandelt.
  \begin{abcenum}
    \item $\cos \theta'=\frac{u_\perp'}{c}$, wobei $u'_\perp$ die Geschwindigkeit des Lichtes in $\Sigma'$ senkrecht zur Milchstraße ist.  Es gilt die rel. Geschwindigkeitsaddition:
      \begin{equation*}
        u_\perp'=\frac{u_\perp-v}{1-\frac{u_\perp v}{c^2}}
      \end{equation*}
  Wobei $u_\perp$ die Geschwindigkeit des Lichtes im System $\Sigma$ senkrecht zur Milchstraße ist und $v$ die Geschwindigkeit des Raumschiffes.  Nun ergibt sich:
  \begin{align*}
    \cos\theta'&=\frac{\cos\theta-\beta}{1-\beta\cos\theta}; u_\perp=c\cos\theta; u_\perp'=c\cos\theta'; \beta=\frac vc\\
\theta'=\arccos\left(\frac{\cos\theta -\beta}{1-\beta\cos\theta}\right)
  \end{align*}
    \item
      \begin{align*}
        \cos\theta'(1-\beta\cos\theta)&=\cos\theta-\beta\\
        \cos\theta'-\cos\theta &= \beta(\cos\theta\cos\theta' -1)\\
        \beta &= \frac{\cos\theta-\cos\theta'}{1-\cos\theta\cos\theta'}
      \end{align*}
      $\cos\theta'=\cos 150\unit{\cel}=-\cos 30\unit{\cel}=-\frac {\sqrt{3}}2$, $\cos\theta=\frac{20000\ly}{\frac 12 100000\ly}=\frac 25$.
      \begin{equation*}
        \beta=\frac{\frac 25+\frac{\sqrt{3}}2}{1+\frac{\sqrt{3}}5}=\frac{\frac 25+\frac{\sqrt{3}}2-\frac{2\sqrt{3}}{25}-\frac 3{10}}{1-\frac 3{25}}=\frac{10-2\sqrt{3}-\frac{15}2+\frac{25\sqrt{3}}2}{22}=\frac{5+21\sqrt{3}}{44}
      \end{equation*}
  
  \end{abcenum}
  Somit ergibt sich $v=\frac{5+21\sqrt{3}}{44}c$.  Da $3\geq\frac{169}{49}=3+\frac{22}{49}$ ist $v<c$.
\end{solution}
\end{problem}

\input{previewending.tex}
