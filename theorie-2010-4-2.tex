\ifx \mpreamble \undefined
\documentclass[12pt,a4paper]{article}
\usepackage{answers}
\usepackage{microtype}
\usepackage[left=3cm,top=2cm,bottom=3cm,right=2cm,includehead,includefoot]{geometry}

\usepackage{amsfonts,amsmath,amssymb,amsthm,graphicx}
\usepackage[utf8]{inputenc}
\usepackage[T1]{fontenc}
\usepackage{ngerman}

\usepackage{pstricks}
\usepackage{pst-circ}
\usepackage{pst-plot}
%\usepackage{pst-node}
\usepackage{booktabs}

\ifx \envfinal \empty
\usepackage{pst-pdf}

\usepackage{hyperref}
\hypersetup{
bookmarks=true,
pdfpagemode=UseNone,
pdfstartview=FitH,
pdfdisplaydoctitle=true,
pdflang=de-DE,
pdfborder={0 0 0}, % No link borders
unicode=true,
pdftitle={IPhO-Aufgabensammlung},
pdfauthor={Pavel Zorin},
pdfsubject={Aufgaben der 3. und 4. Runden deutscher Auswahl zur IPhO},
pdfkeywords={}
}

\fi

%Times 10^n
\newcommand{\ee}[1]{\cdot 10^{#1}}
%Units
\newcommand{\unit}[1]{\,\mathrm{#1}}
%Differential d's
\newcommand{\dif}{\mathrm{d}}
\newcommand{\tdif}[2]{\frac{\dif#1}{\dif#2}}
\newcommand{\pdif}[2]{\frac{\partial#1}{\partial#2}}
\newcommand{\ppdif}[2]{\frac{\partial^{2}#1}{\partial#2^{2}}}
%Degree
\newcommand{\degr}{^\circ}
%Degree Celsius (C) symbol
\newcommand{\cel}{\,^\circ\mathrm{C}}
% Hinweis
\newcommand{\hinweis}{\emph{Hinweis:} }
% Aufgaben mit Buchstaben numerieren
\newenvironment{abcenum}{\renewcommand{\labelenumi}{(\alph{enumi})} \begin{enumerate}}{\end{enumerate}\renewcommand{\labelenumi}{\theenumi .}}

\def \mpreamble {}
\else
%\ref{test}
\fi
\ifx \envfinal \undefined


\newcommand{\skizze}[1]{
\begin{figure}
\begin{center}
#1
\end{center}
\end{figure}
}




%\documentclass[12pt,a4paper]{article}
\renewcommand{\thesection}{}
\renewcommand{\thesubsection}{\arabic{subsection}}

\newenvironment{problem}[2]{
\subsection{#1 \emph{(#2 Punkte)}}
}{}
\newenvironment{solution}{\subsection{Lösung}}{}
\newenvironment{expsolution}{\subsection*{Lösung}}{}

\begin{document}

\fi

\begin{problem}{Warme Kugel}{4}
Punktlichtquelle A und eine Kugel mit Radius $R$ haben den Abstand $d$ und die Kugel die Gleichgewichtstemperatur $T_1=20\cel$ bei Bestrahlung durch die Punktlichtquelle.\\
Wenn man nun eine Linse in die Mitte mit Radius $2R$ und der Brennweite $d/3$ packt, verändert sich die Temperatur, bestimmen Sie diese! (Nehmen Sie $R<<d$ an)
\end{problem}

\begin{problem}{Bestrahltes Prisma}{5}
Gegeben sei ein Prisma der Breite der Grundseite $a$, dem Öffnungewinkel $\alpha=120\degr$, dem Brechungsindex $n=1.5$. Von oben strahlt eine Lichtleistung $P_S$ mit der Breite $\frac{3}{4} a$ auf das Prisma. Das Prisma ist in einer horizontalen Achse beweglich (Die Achse in der die Breite der Grundseite und der Strahlung definiert wurde).
\begin{abcenum}
\item Bestimmen Sie die auf das Prisma wirkende Kraft in Abhängigkeit vom Ort auf dieser Achse.
\end{abcenum}
\end{problem}

\begin{problem}{Space-Taxi}{4.5}
Ein Space-Taxi ist auf einer Umlaufbahn um die Erde (Masse $M=6\cdot 10^{24}\,kg$) mit Radius $r_1=2\cdot 10^4\,km$. Es soll von Punkt A startend eine Raumstation im Punkt B (Bahnradius $r_2=2r_1$) erreichen, wobei B auf der gegenüberliegenden Seite relativ zur Erde liegt.
\begin{abcenum}
\item Bestimmen Sie die nötigen Geschwindigkeitsänderungen des Space-Taxis in A und B und auf die äußere Umlaufbahn zu kommen.
\item Bestimmen Sie den Winkel, den die Raumstation zum Startpunkt A des Taxis (relativ zur Erde) haben muss, damit diese gleichzeitig bei B sind.
\end{abcenum}
\end{problem}

\begin{problem}{Ideale Klimaanlage}{4.5}
Gegeben ist eine Klimaanlage, wobei im Haus eine Temperatur von $T_i$ und außen von $T_a$ vorliegt. Die Klimaanlage hat die Leistung $P_{el}$
\begin{abcenum}
\item Bestimmen Sie die maximal mögliche Effizienz der Klimaanlage
\item Bei 30\% der Leistung ist die Gleichgewichtstemperatur $T_i=20\unit\cel$ bei $T_a=30\unit\cel$. Gegeben ist noch die Information, das der Wärmedurchgang durch die Wand proportional zur Temperaturdifferenz ist. Wie groß kann $T_a$ maximal werden, wenn die Leistung zu 100\% ausgenutzt wird.
\item Die Klimaanlage arbeitet als Heizung. Wie kalt kann es draußen sein, wenn $T_i=20\cel$ sein soll?
\end{abcenum}
\end{problem}


\begin{problem}{Fliegender Ring}{6.5}
Ein Ring (Radius $R$, Masse $m$) mit einem Stromfluss $I$ befindet sich im Abstand $d$ ($d<<R$) über eine unendlichen, widerstandsfrei leitenden Ebene (auf der Oberfläche ist Magnetfeld $B=0$).
\begin{abcenum}
\item Bestimmen Sie $d$, bei der sich der Ring im Gleichgewicht befindet.
\item Bestimmen Sie die Schwingunsfrequenz des Ringes für kleine Auslenkungen aus der Ruhelage.
\end{abcenum}
\end{problem}

%\begin{problem}{Rotierender Stab}{5.5}
%Zeichung notwendig
%\end{problem}

\input{previewending.tex}

