\ifx \headertex \undefined
\documentclass[11pt,a4paper,oneside,openany]{memoir}
\usepackage{answers}
\usepackage{microtype}
\usepackage[left=3cm,top=2cm,bottom=3cm,right=2cm,includehead,includefoot]{geometry}

\usepackage{amsfonts,amsmath,amssymb,amsthm}
\usepackage[utf8]{inputenc}
\usepackage[T1]{fontenc}
\usepackage[ngerman]{babel}

\usepackage{pstricks}
\usepackage{pst-circ}
\usepackage{pst-plot}
%\usepackage{pst-node}
\usepackage{booktabs}

%Times 10^n
\newcommand{\ee}[1]{\cdot 10^{#1}}
%Units
\newcommand{\unit}[1]{\,\mathrm{#1}}
%Differential d's
\newcommand{\dif}{\mathrm{d}}
\newcommand{\tdif}[2]{\frac{\dif#1}{\dif#2}}
\newcommand{\pdif}[2]{\frac{\partial#1}{\partial#2}}
\newcommand{\ppdif}[2]{\frac{\partial^{2}#1}{\partial#2^{2}}}
%Degree
\newcommand{\degr}{^\circ}
%Degree Celsius (C) symbol
\newcommand{\cel}{\unit{^\circ C}}
% Hinweis
\newcommand{\hinweis}{\emph{Hinweis:} }
% Aufgaben mit Buchstaben numerieren
\newenvironment{abcenum}{\renewcommand{\labelenumi}{(\alph{enumi})} \begin{enumerate}}{\end{enumerate}\renewcommand{\labelenumi}{\theenumi .}}

\setsecnumdepth{none}

\ifx \envfinal \undefined % Preview mode
\newcommand{\skizze}[1]{
\begin{figure}
\begin{center}
#1
\end{center}
\end{figure}
}

\renewcommand{\thesection}{}
\renewcommand{\thesubsection}{\arabic{subsection}}

\newenvironment{problem}[2]{
\subsection{#1 \emph{(#2 Punkte)}}
}{}
\newenvironment{solution}{\subsection*{Lösung}}{}
\newenvironment{expsolution}{\subsection*{Lösung}}{}

\else % Final mode

\usepackage{pst-pdf}

\usepackage[bookmarks=true]{hyperref}
\hypersetup{
pdfpagemode=UseNone,
pdfstartview=FitH,
pdfdisplaydoctitle=true,
pdflang=de-DE,
pdfborder={0 0 0}, % No link borders
unicode=true,
pdftitle={IPhO-Aufgabensammlung},
pdfauthor={Pavel Zorin},
pdfsubject={Aufgaben der 3. und 4. Runden der deutschen Auswahl zur IPhO},
pdfkeywords={}
}

\newcommand{\skizze}[1]{
\begin{center}
#1
\end{center}
}

%%%%%%%%%%%%%%%%%%%%%%%%%%%%%%%%%%%%%%%%%%%%%%%%%%%%%%%%%%%%%%%%%%%%%%%%%%%%%%%%%%%%%%%%%%%
%%%%%%%%%%%%%%%%%%%%%%%%%%%%%%%%%%%%%%%%%%%%%%%%%%%%%%%%%%%%%%%%%%%%%%%%%%%%%%%%%%%%%%%%%%%
%
%     Formatierung der Aufgaben/Lösungen
%
%     Anmerkung dazu: nicht-ASCII-Zeichen in Überschriften gehen aus rätselhaften Gründen
%     nicht. Sie werden zwar bei der Aufgabe richtig angezeigt, in die Lösungsdatei wird
%     aber eine unverständliche Sequenz geschrieben, die dann nicht wieder gelesen werden
%     kann.
%
%%%%%%%%%%%%%%%%%%%%%%%%%%%%%%%%%%%%%%%%%%%%%%%%%%%%%%%%%%%%%%%%%%%%%%%%%%%%%%%%%%%%%%%%%%%
%%%%%%%%%%%%%%%%%%%%%%%%%%%%%%%%%%%%%%%%%%%%%%%%%%%%%%%%%%%%%%%%%%%%%%%%%%%%%%%%%%%%%%%%%%%
\newcounter{problem}

\newenvironment{problem}[2]{
\addtocounter{problem}{1}
\subsection{Aufgabe \arabic{problem}: #1 \emph{(#2 Punkte)}}
\renewcommand{\Currentlabel}{<\arabic{problem}><#1>}
}{}
\Newassociation{solution}{Soln}{solutions}
\Newassociation{expsolution}{Soln}{expsolutions}
\def\solTitle<#1><#2>{
\subsection{Aufgabe #1: #2}
}
\renewenvironment{Soln}[1]{
\solTitle #1
}{}

\fi

\def \headertex {}
\begin{document}
\fi


\begin{problem}{Insektenverfolgung}{3,5}
Eine Fledermaus fliegt mit $v_F=25\unit{\frac{km}{h}}$ in einem Medium mit einer Schallgeschwindigkeit von $c=340\unit{\frac{m}{s}}$.
Sie sendet ein Signal mit der Frequenz $f_1=40\unit{kHz}$ aus, welches an einem Insekt reflektiert wird. Die Fledermaus empfängt das reflektierte Signal mit einer Frequenz von $f_2=40.4\unit{kHz}$. Bestimmen Sie die Geschwindigkeit $v_I$ des Insekts.
\begin{solution}
Die Fledermaus empfängt wegen des Dopplereffekts das Signal $f_2$ wie folgt:
\[
f_2=f_1\cdot \left(\dfrac{1+\frac{v_F}{c}}{1+\frac{v_I}{c}}\right)\cdot \left(\dfrac{1-\frac{v_I}{c}}{1-\frac{v_F}{c}}\right)=f_1\cdot \left(1+\dfrac{2c\cdot \Delta v}{c^2-v_F\cdot v_I-c\cdot \Delta v}\right)
\]
Dies ergibt $v_I\approx 18.9\unit{\frac{km}{h}}$.
\end{solution}
\end{problem}

\begin{problem}{Batterieschaltung}{4}
Eine Lampe soll unter Spannung $U=10\unit{V}$ betrieben werden. Ihre Leistung beträgt dabei $P_L=15\unit{W}$.
Wie viele Batterien mit dem Innenwiderstand $R_i=2\unit{\Omega}$ und der Spannung $U_0=1.5\unit{V}$ sind dafür mindestens notwendig? Geben Sie eine Schaltung mit möglichst wenig Batterien an.
\begin{solution}
Die effektive Leistung einer Batterie ist
\[
P_B=I^2\cdot R=\left(\dfrac{U_0}{R+R_i}\right)^2\cdot R=\dfrac{U_0^2\cdot R}{R^2+2R_i\cdot R+R_i^2}.
\]
Diese wird maximiert wenn die Ableitung nach $R$ verschwindet. Dies ist bei $R=R_{i}$ der Fall.
Also ist
\[
P_{max}=\dfrac{U_0^2}{4R_i}\approx 0.281 \unit{W}
\]
Für die minimale Zahl ergibt sich also $N_{min}=\lceil\dfrac{P_L}{P_{max}}\rceil=54$.\\
Eine mögliche Schaltung (nicht minimal, war aber auch nicht gefordert) ist:
Reihenschaltung von 10 Schaltungen, die jeweils 6 parallelgeschaltete Batterien enthalten. 60 Batterien.
\end{solution}
\end{problem}

\begin{problem}{Faserkreisel}{4,5}
Gegeben ist ein kreisförmiger Glasfaserring mit Radius R, der sich mit der Winkelgeschwindigkeit $\omega$ im Uhrzeigersinn dreht.
An einer Stelle dringt Licht mit Wellenlänge $\lambda$ ein und geht in beide Richtungen.
\begin{abcenum}
\item Wie groß ist die Phasendifferenz nach dem Umlaufen von N Windungen am Startpunkt.
\item Wieviele Windungen benötigt ein Faserkreisel am Nordpol mit $R=1\,m$ und $\lambda= 655\,nm$ (Tageslänge $24\,h$), damit destruktive Interferenz verursacht wird.
\end{abcenum}
\end{problem}

\begin{problem}{Rollende Kegel}{7}
Gegeben ist ein Doppelkegel mit Öffnungswinkel $\alpha$ an der Spitze, der Masse M und dem Radius $R$ der Grundseite. Er liegt am Anfang A  von zwei langen Stangen, die einen Winkel $\gamma$ zum Boden und einen Winkel $\beta$ untereinander besitzen. Beide Stangen beginnen bei A.
\begin{abcenum}
\item Unter welchen Umständen rollt der Zylinder in A los.
\item Bestimmen Sie das Trägheitsmoment des Doppelzylinders.
\item Gebe die Geschwindigkeit in Abhängigkeit vom Ort an.
\item Wie weit rollt der Kegel? Bestimme die maximale Geschwindigkeit.
\end{abcenum}
\end{problem}

\begin{problem}{Brennstoffzelle}{11}
\textit{Diese Aufgabe wird niemals nachgeliefert, da sie sehr lang und vor allem hässlich ist.}
\end{problem}

\ifx \envfinal \undefined
\end{document}
\fi